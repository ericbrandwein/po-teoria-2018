\documentclass{beamer}

\usepackage[utf8]{inputenc}
\usepackage[spanish]{babel}
% \usepackage{comment}

\usetheme{Frankfurt}

%Information to be included in the title page:
\title{Prueba de Oposición}
\subtitle{Área Teoría}
\institute{Universidad de Buenos Aires, FCEyN}
\author{Eric Brandwein}
\date{5 de Noviembre de 2018}

\begin{document}

\frame{\titlepage}

\section{Marco del Ejercicio}
\begin{frame}
\frametitle{Marco del Ejercicio}

Materia: Paradigmas de Lenguajes de Programación

\end{frame}

\begin{frame}
\frametitle{Marco del Ejercicio}
Guías prácticas:
\begin{itemize}
    \item 0: Repaso de programación funcional
    \item 1: Programación funcional
    \item 2: Cálculo Lambda Tipado
    \item \textbf{3: Inferencia de Tipos}
    \item 4: Subtipado
    \item 5: Programación orientada a objetos
    \item 6: Programación lógica
    \item 7: Resolución en lógica
\end{itemize}
\end{frame}


\begin{frame}
\frametitle{Marco del Ejercicio}

\begin{itemize}
    \item Los alumnos deben conocer los \textbf{términos}
        y el sistema de \textbf{juicios de tipado}
        de $\lambda^{bn}$.
    \item El ejercicio puede ser dado:
    \begin{itemize}
        \item en la clase en la que se está enseñando
            $Erase()$ por \textbf{primera vez}.
            % Por ser un ejercicio simple
        \item como uno de los \textbf{primeros ejercicios} de
            la práctica 3.
    \end{itemize}
\end{itemize}
\end{frame}

\section{Objetivos del ejercicio}
\begin{frame}
\frametitle{Objetivos del ejercicio}
\begin{itemize}
    \item Entender el \textbf{funcionamiento} del $Erase()$.
    \item Repasar qué son las \textbf{anotaciones de tipo},
        y cuándo y cómo afectan al tipo del término.
    \item Hacer notar que \textbf{no es suficiente} el
        $Erase()$ de un término para determinar su tipo.
\end{itemize}
\end{frame}


\section{¿Por qué este ejercicio?}
\begin{frame}
\frametitle{¿Por qué este ejercicio?}
\begin{itemize}
    \item Es un ejercicio \textbf{simple},
        de fácil entendimiento.
    \item Tiene potencial de \textbf{participación}
        de parte de los alumnos.
\end{itemize}

\end{frame}

\section{Enunciado}
\begin{frame}
\frametitle{Enunciado}
Mostrar, si es posible, dos términos
$M_1 \neq M_2 \in \lambda^{bn}$ tal que
$\emptyset \triangleright M_1 : \sigma$,
$\emptyset \triangleright M_2 : \sigma'$, $\sigma \neq \sigma'$
y $Erase(M_1) = Erase(M_2)$.
\end{frame}

\begin{frame}
\frametitle{Términos de $\lambda^{bn}$}

\qquad\qquad\qquad\qquad$M, P, Q ::= true$

\qquad\qquad\qquad\qquad\qquad\qquad$\ \ |\ false$

\qquad\qquad\qquad\qquad\qquad\qquad$\ \ |\ if\ M\ then\ P\ else\ Q$

\qquad\qquad\qquad\qquad\qquad\qquad$\ \ |\ M\ N$

\qquad\qquad\qquad\qquad\qquad\qquad$\ \ |\ \lambda x : \sigma.M$

\qquad\qquad\qquad\qquad\qquad\qquad$\ \ |\ x$

\qquad\qquad\qquad\qquad\qquad\qquad$\ \ |\ 0$

\qquad\qquad\qquad\qquad\qquad\qquad$\ \ |\ succ(M)$

\qquad\qquad\qquad\qquad\qquad\qquad$\ \ |\ pred(M)$

\qquad\qquad\qquad\qquad\qquad\qquad$\ \ |\ iszero(M)$

\end{frame}
\end{document}
